\ProvidesFile{Intro.tex}[Введение]


\section{Несколько общих слов}

\paragraph{Пару слов о себе}

Как я уже заявил выше, я не программист, я -- математик.
У этого есть несколько важных импликаций, которые стоит понимать:
\begin{enumerate}
\item У меня другой взгляд на программирование.
Так как я все же рос и развивался в среде математиков и в программирование пришел сам и по своей воле, то я уже обладал достаточной профдеформацией, что скорее всего моя точка зрения будет сильно отличаться от того, что вы видите и слышите вокруг.
Это наверное самое ценное, чем я могу поделиться с вами в идейном плане, ибо новые идеи обычно брать неоткуда.
Надеюсь, что то что вижу и замечаю я окажется для вас полезным.

\item Отсутствие опыта в индустрии означает сразу две вещи:
\begin{enumerate}
\item Отсутствие опыта и знакомства со стандартными инструментами разработки.
Это моя самая слабая часть.
Но тут я все же наверстываю.

\item Отсутствие вредных привычек.
Когда вы пишите код на работе у вас обычно куча дедлайнов и самая главная цель -- получить результат в установленный срок.
При этом у вас обычно есть какая-то команда с разным бэкграундом, есть какое-то legacy и проблемы с документацией к коду.
В итоге приходится писать так как получается, лишь бы работало.
Не всегда есть время подумать над тем как лучше.
В моем случае я всегда неограничен во времени над тем, чтобы подумать как решать ту или иную проблему при написании кода.
\end{enumerate}
\end{enumerate}

\paragraph{Как ориентироваться?}

Надо понимать, что какой бы признак вы ни выделили среди людей, реально крутых людей с данным признаком будет мало.
Навык программирования (в широком смысле) не исключение.
Потому большинство того, что вы видите, слышите и читаете -- это мусор.
Со временем вы конечно разберетесь с этим вопросом, но пока нет достаточного опыта, то тут важно смотреть на ключевых игроков индустрии и понять, как на индустрию смотрят они, что они предлагают, какие проблемы видят и в целом перенимать опыт у ключевых фигур -- очень важно.
Вот для ориентира некоторый список людей, на которых я бы обратил внимание
\begin{enumerate}
\item Andrei Alexandrescu

\item Sean Parent

\item Scott Mayers
\end{enumerate}
Так же следите за разными конферейнциями по C++, из которых самая известная пожалуй cppcon.
Правда и на ней выступают убогие с мусорными докладами или что хуже с вредящими докладами.
Я бы мог добавить к списку выше еще с десяток приличных людей, чьи взгляды и мнение повлияли на меня, но я думаю, что этих я бы выделил отдельно.
Буду ли я расширять этот список или добавлю ли еще кого-то, посмотрим.

\paragraph{Про что этот текст?}

Умение проектировать программы и писать хороший код -- это умение, которому тоже надо учиться.
Написать так, чтобы просто работало -- этого мало.
Потому я хочу посвятить этот текст следующим вопросам:
\begin{enumerate}
\item Как писать хороший код.
Что значит хороший код.
Какие есть метрики или способы сравнивать подходы для написания кода.

\item Архитектура и дизайн.
Какие приемы, шаблоны и техники стоит использовать и почему, а какие лучше не вспоминать.

\item Языковые особенности.
Какие тонкости языка вам надо знать, чтобы не огрести от него.
Язык C++ очень плохой и очень тяжелый, схлопотать undefined behavior можно запросто независимо от вашего опыта, а компилятор никогда ничего вам не должен и не поможет ни в чем.
Потому важно обращать внимание на важные детали языка и быть готовыми к возможным проблемам.
\end{enumerate}

\paragraph{Чтобы почувствовать стиль изложения}

Я хочу начать с двух аксиом:
\begin{enumerate}
\item С++ -- говно.

\item Для меня это самый комфортный для использования язык.
\end{enumerate}
Надо понимать, что язык действительно очень плохой и архаичный.
С плохой экосистемой, плохими решениями на очень разных уровнях.
Стихийно развивавшийся и обросший кучей противоречивых костылей.
А самое главное -- никакой помощи со стороны компилятора.
А чем хуже язык -- тем больше вам понадобится помощь со стороны инструментов разработки.
Так же важно понимать, что чем больше в языке ограничений -- тем лучше.
Чем меньше у программиста свободы, тем сложнее написать код плохо.
В C++ к сожалению свобода программиста безгранична.
Это компилятор Rust-а умрет прежде чем пропустит вашу ошибку, потом прочитает вам лекцию почему так делать не надо, как это исправить и будет долго и упорно вести вас к верному решению.
В случае C++ мы можем лишь верить, что наше приключение не закончится так и не начавшись.
